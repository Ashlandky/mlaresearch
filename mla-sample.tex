% !TEX encoding = UTF-8 Unicode
\documentclass[russian,english]{mlaresearch}

% Should test for XeLaTeX here
\usepackage{fontspec}
\usepackage{polyglossia}
\setdefaultlanguage{english}
\setotherlanguage{russian}

% Times on OS X doesn't have small caps so substitute
% \setmainfont{texgyretermes-regular.otf}
\setmainfont{Times New Roman}[SmallCapsFont=TeX Gyre Termes,SmallCapsFeatures={Letters=SmallCaps}] 
\setsansfont{Century Gothic}
\setmonofont{Courier New}



\addbibresource{crime.bib}


\title{Rocky Raskolnikov: Singing \textit{Rocky Raccoon} with a Russian accent}
\author{Jeffrey Goldberg}
\lastname{Goldberg}
\course{Adv.\ Lit\. BS}
\instructor{Morris Zapp}
\date{7 March, 2014}

\begin{document}
\makemlatitle

Perhaps someone suffering the delirium of fever and hunger might see the
Beatles song \textit{Rocky Raccoon} \autocite{Beatles:RR} as a retelling of
\textit{Crime and Punishment} \autocite{Dostoyevsky:CP}, but my descent into
madness has not taken me that far. However, there are intriguing connections
between the two that are worth exploring. These will be explored in the
sections that follow, finally concluding with some comments on the nature of
coincidence between literary works.

\section{Only to find Sonia's Bible}

The final verse of \autocite{Beatles:RR} tells of Rocky Raccoon's, the song's protagonist, unlikely salvation:

\begin{verse}
Rocky Raccoon fell back into his room\\
Only to find Gideon's Bible\\
The Gideons checked out\\
And left it no doubt\\
To aid in poor Rocky's revival
\end{verse}

This seemingly bizarre turn of events (a religious conversion and redemption
of a someone who attempted to kill a sexual rival) is reminiscent of the
unusual ending, presented in the epilogue, of \textit{CP}, its protagonist,
Rodion Romanovich Raskolnikov, is an intellectual who is contemptuous of
religion throughout the story; but Rodion finds redemption after finding a
Bible left in his bed by Sofia Semyonovna Marmeladova (Sonja), a young women
of simple faith.

[more stuff about why this is an odd ending in both cases]

\section{Incompetent crimes}

Both Rocky and Rodion plan their crimes in advance, and make a number of preparations. Rocky travels some distance to track down his intended victim, Danial, and has come with a weapon. Rodion has been contemplating his murder for more than one month, makes a trial visit to his victim's apartment, and has prepared a method to conceal his weapon. Yet neither attack goes as planned.

Rocky losses and old west style quick draw. “But Danial was hot; he drew first and shot” \autocite{Beatles:RR}. Rodion succeeds at killing his intended victim, but among other things, he ends up killing the victim's sister when she comes across the scene, finds himself trapped in the room with two bodies, fails to steal the cash he was after, and manages to avoid detection through a series of nearly farcical events that had nothing to do with is extensive planning and preparation. Although Rodion succeeds at the murder and avoids immediate detections, he still has bungled it. He also fails to achieve any of his self-proclaimed goals for committing the crime.

\section{A coincidence of names}

Both Rocky Raccoon and Rodion Romanovich Raskolnikov have all components of their names beginning with the letter ‘R’: Rocky with two, and Rodion with three. Following Russian tradition, Rodion is often referred to by his given name along with his patronym, thus “Rodion Romanovich”.

This leads to an a point that all English speakers have when reading Russian literature. The variety of names that a person can be referred to or called is overwhelming. Sofia Semyonovna Marmeladova is almost never referred to as “Sofia” but is instead called “Sonia” or “Sonjechka”. This is echoed in \textit{Rocky Raccoon} with “Her name was McGill, she called herself Jill, but everyone knew her as Nancy”.



\printbibliography

\end{document}  
